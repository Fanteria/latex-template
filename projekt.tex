\documentclass[a4paper,12pt]{article}
\usepackage[utf8]{inputenc}
\usepackage[czech]{babel}
\usepackage[T1]{fontenc}
\usepackage[left=3.5cm,right=2cm,top=3cm,bottom=3cm]{geometry}
\usepackage{amsmath,amsfonts,amssymb}
\usepackage{enumerate}
\usepackage{gensymb,marvosym}
\usepackage{times}
\usepackage{tabularx}
\usepackage{graphicx}
\usepackage{fancyhdr}
%\usepackage{pdfpages}

\makeatletter

\def\title#1{\gdef\inserttitle{#1}\gdef\@title{#1}\hypersetup{pdftitle={#1}}}
\def\author#1{\gdef\insertauthor{#1}\gdef\@author{#1}\hypersetup{pdfauthor={#1}}}
\def\supervisor#1{\gdef\insertsupervisor{#1}}
\def\institution#1{\gdef\insertinstitution{#1}\gdef\@institution{#1}\hypersetup{pdfproducer={#1}}}
\def\faculty#1{\gdef\insertfaculty{#1}}
\def\department#1{\gdef\insertdepartment{#1}}
\def\location#1{\gdef\insertlocation{#1}}
\def\papertype#1{\gdef\insertpapertype{#1}}
\def\subject#1{\gdef\insertsubject{#1}\hypersetup{pdfsubject={#1}}}
\def\keywords#1{\gdef\insertkeywords{#1}\hypersetup{pdfkeywords={#1}}}

\makeatother

\usepackage[
  backend=biber,     % defaultní možnost, nastaví unicode a několik dalších vlastností
  style=iso-numeric, % iso-numeric pro číselné uspořádnání nebo iso-authoryear pro uspořádání pomocí autorů
]{biblatex}

\addbibresource{literatura.bib}

% Více řádků v jednom pro tabulku
\usepackage{multirow}

% Pro tečkovanou čáru pro podpis
\usepackage{arydshln}

\usepackage[none]{hyphenat} \sloppy
\clubpenalty 10000
\widowpenalty 10000

% Nastavení řádkování
\usepackage{setspace} \onehalfspacing

% Nastavení prolinkování odkazů v dokumentu
\usepackage{hyperref}
\hypersetup{
    pdfcreator = {\LaTeX\ with\ Bib\LaTeX},
    colorlinks = false,
    hidelinks
}


% Nastavení cesty k obrázkům
\graphicspath{{pics/}}

% Funkce pro vkládání grafů
\usepackage{float}
\newfloat{graf}{hbtp}{ext}
\floatname{graf}{Graf}

% Pojmenovaní obrázku
\usepackage{caption}
\captionsetup[figure]{name=Obr.}

% === Nastavení proměnných ===
\title{Název práce}
\author{Autor práce}
\supervisor{Vedoucí práce}
\institution{Škola}
\faculty{Fakulta}
\department{Obor}
\location{Místo vytvoření}
\papertype{Typ práce}
\subject{Obsah dokumentu}
\keywords{Klíčová slova}

% === Začátek dokumentu ===
\begin{document}
\pagestyle{empty}

\begin{titlepage}
	\centering

  \vfill

	{\LARGE \insertinstitution \par}
	{\Large \insertfaculty \par}

	\vfill

	{\huge\bfseries \inserttitle \par}
	{\Large \insertpapertype \par}

  \vfill

  {\Large Vedoucí práce: \insertsupervisor \par}
	{\Large Autor práce: {\insertauthor} \par}

	\vspace{1.5cm}

	{\scshape\large \insertlocation ~ \the\year \par}

  \vfill
\end{titlepage}


% Vložení stránky se zadáním
\newgeometry{left=0cm,right=0cm,top=0cm,bottom=0cm}
\includegraphics{../zadani.pdf}
\restoregeometry
\clearpage


\vspace*{\fill}
\section*{Čestné prohlášení}
Prohlašuji, že jsem diplomovou/bakalářskou práci na téma: \nazev vypracoval/a samostatně a použil/a jen pramenů, které cituji a uvádím v seznamu použitých zdrojů.

Jsem si vědom/a, že odevzdáním bakalářské práce souhlasím s jejím zveřejněním dle zákona č. 111/1998 Sb., o vysokých školách a o změně a doplnění dalších zákonů, ve znění pozdějších předpisů, a to i bez ohledu na výsledek její obhajoby.

Jsem si vědom/a, že moje bakalářská práce bude uložena v elektronické podobě v univerzitní databázi a bude veřejně přístupná k nahlédnutí.

Jsem si vědom/a že, na moji bakalářskou práci se plně vztahuje zákon č. 121/2000 Sb., o právu autorském, o právech souvisejících s právem autorským a o změně některých zákonů, ve znění pozdějších předpisů, především ustanovení § 35 odst. 3 tohoto zákona, tj. o užití tohoto díla.

\qquad

\setlength{\dashlinedash}{1pt}
\setlength{\dashlinegap}{1pt}
\setlength{\arrayrulewidth}{1pt}

\noindent
\begin{tabularx}{\textwidth}{X r}
V Praze dne \today &
\begin{tabular}[b]{@{} p{6cm} @{}}
\\
\hdashline
\centering
podpis
\end{tabular}
\\
\end{tabularx}


\include{podekovani}

% === Nastavení obsahů pro dokumenty ===
% tabulka s obsahem
\tableofcontents

% seznam zkratek
% pouze u většího množství zkratek
%\include{seznamZkratek}

% seznam tabulek
% pouze pokud jsou použity více jak tři tabulky
%\newpage
%\listoftables
%\thispagestyle{empty}

% seznam obrázků
% pouze pokud jsou použity více jak tři obrázky
%\newpage
%\listoffigures
%\thispagestyle{empty}

\setcounter{page}{1} %cislo strany
\thispagestyle{empty}
%\pagestyle{empty}

\newpage

\pagestyle{fancy} %vlastni zahlavi/zapati
\renewcommand{\headrulewidth}{0.5pt} %bez linky v zahlavi
\renewcommand{\footrulewidth}{0pt} %linka v zapati
\lhead{\leftmark}       \chead{} \rhead{} %pole zahlavi (prazdna)
\lfoot{} \cfoot{} \rfoot{\thepage} %pole zapati


\section{Some section}
Text from input file. Here is some xyz in that file xyz xyz xyz xyz xyz.


% === Zdroje ===
\clearpage
\phantomsection % přidání odkazu do PDF záložek
\addcontentsline{toc}{section}{Seznam použitých zdrojů}
\renewcommand{\refname}{Seznam použitých zdrojů}

\printbibliography

\end{document}
