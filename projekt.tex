\documentclass[a4paper,12pt]{article}
\usepackage[utf8]{inputenc}
\usepackage[czech]{babel}
\usepackage[T1]{fontenc}
\usepackage[left=3.5cm,right=2cm,top=3cm,bottom=3cm]{geometry}
\usepackage{amsmath,amsfonts,amssymb}
\usepackage{enumerate}
\usepackage{gensymb,marvosym}
\usepackage{times}
\usepackage{environ}
\usepackage{tabularx}
\usepackage{graphicx}
\usepackage{fancyhdr}
\usepackage{xparse}
%\usepackage{pdfpages}

\ExplSyntaxOn

\cs_new_protected:Npn \pirooh_call_with_one_arg:Nnnnn #1#2#3#4#5 {
  #1 {#2} {#3} {#4} {#5}
}

\cs_generate_variant:Nn \pirooh_call_with_one_arg:Nnnnn { Nxxxx }

\tl_new:N \l__is_instring_tl

\cs_new_protected:Nn \comm_instring:nnnn {
  \tl_set:Nn \l__is_instring_tl {#1}
  \regex_match:nnTF {\u{l__is_instring_tl}}{#2}{#3}{#4}
}

\NewDocumentCommand{\callWithExpandedArg}{mmmmm} {
  \pirooh_call_with_one_arg:Nxxxx #1 {#2} {#3} {#4} {#5}
}

\NewDocumentCommand{\iscontainsunexpanded}{mmmm} {
  \comm_instring:nnnn {#2}{#1}{#3}{#4}
}

\NewDocumentCommand{\iscontains}{mmmm} {
  \callWithExpandedArg{\iscontainsunexpanded}{#1}{#2}{#3}{#4}
}

\NewDocumentCommand{\isempty}{mmm} {
  \ifthenelse{\equal{#1}{}} {#2} {#3}
}

\NewDocumentCommand{\printsupervisor}{m} {
  \isempty{#1} {} {Vedoucí~práce:~{#1}}
}

\NewDocumentCommand{\printautor}{m} {
  \isempty{#1}
  {}
  {\iscontains{#1}{,}
    {Autoři~práce:~{#1}}
    {Autor~práce:~{#1}}
  }
}

\ExplSyntaxOff

\newcommand{\insertpicture}[4][]{
\begin{figure}[htbp]
  \centering
  \includegraphics[width={#3}]{#2}
  \caption{#4}
  \isempty{#1}{}{\label{#1}}
\end{figure}
}

\newcommand{\insertgraph}[4][]{
\begin{figure}[htbp]
  \centering
  \includegraphics[width={#3}]{#2}
  \captionof{graf}{#4}
  \isempty{#1}{}{\label{#1}}
\end{figure}
}

\NewEnviron{inserttable}[3][]{
\begin{centering}
  \begin{tabular}{#2}
    \BODY
  \end{tabular}
\captionof{table}{#3}
\isempty{#1}{}{\label{#1}}
\end{centering}
}

\usepackage[
  backend=biber,     % defaultní možnost, nastaví unicode a několik dalších vlastností
  style=iso-numeric, % iso-numeric pro číselné uspořádání nebo iso-authoryear pro uspořádání pomocí autorů
]{biblatex}

\addbibresource{literatura.bib}

% Více řádků v jednom pro tabulku
\usepackage{multirow}

% Pro tečkovanou čáru pro podpis
\usepackage{arydshln}

\usepackage[none]{hyphenat} \sloppy
\clubpenalty 10000
\widowpenalty 10000

% Nastavení řádkování
\usepackage{setspace} \onehalfspacing

% Nastavení prolinkování odkazů v dokumentu t metadat dokumentu
\usepackage{hyperref}
\hypersetup{
    pdftitle={Title}
    pdfauthor={Author}
    pdfproducer={Producer}
    pdfsubject={Subject}
    pdfkeywords={Keywords}
    pdfcreator = {\LaTeX\ with\ Bib\LaTeX},
    colorlinks = false,
    hidelinks
}


% Nastavení cesty k obrázkům
\graphicspath{{pics/}}

% Funkce pro vkládání grafů
\usepackage{float}
\newfloat{graf}{hbtp}{ext}
\floatname{graf}{Graf}

% Pojmenovaní obrázku
\usepackage{caption}
\captionsetup[figure]{name=Obr.}

% === Nastavení proměnných ===
\newcommand*{\doctitle}{Název práce}
\newcommand*{\docauthor}{Autor práce, druhý autor}
\newcommand*{\supervisor}{Vedoucí} %Vedoucí práce
\newcommand*{\institution}{Škola}
\newcommand*{\faculty}{Fakulta}
\newcommand*{\department}{Obor}
\newcommand*{\location}{Místo vytvoření}
\newcommand*{\papertype}{Typ práce}
\newcommand*{\subject}{Obsah dokumentu}
\newcommand*{\keywords}{Klíčová slova}

% === Začátek dokumentu ===
\begin{document}
\pagestyle{empty}

\begin{titlepage}
	\centering

  \vfill

	{\LARGE \insertinstitution \par}
	{\Large \insertfaculty \par}

	\vfill

	{\huge\bfseries \inserttitle \par}
	{\Large \insertpapertype \par}

  \vfill

  {\Large Vedoucí práce: \insertsupervisor \par}
	{\Large Autor práce: {\insertauthor} \par}

	\vspace{1.5cm}

	{\scshape\large \insertlocation ~ \the\year \par}

  \vfill
\end{titlepage}


% Vložení stránky se zadáním
\newgeometry{left=0cm,right=0cm,top=0cm,bottom=0cm}
\includegraphics{../zadani.pdf}
\restoregeometry
\clearpage


\vspace*{\fill}
\section*{Čestné prohlášení}
Prohlašuji, že jsem diplomovou/bakalářskou práci na téma: \nazev vypracoval/a samostatně a použil/a jen pramenů, které cituji a uvádím v seznamu použitých zdrojů.

Jsem si vědom/a, že odevzdáním bakalářské práce souhlasím s jejím zveřejněním dle zákona č. 111/1998 Sb., o vysokých školách a o změně a doplnění dalších zákonů, ve znění pozdějších předpisů, a to i bez ohledu na výsledek její obhajoby.

Jsem si vědom/a, že moje bakalářská práce bude uložena v elektronické podobě v univerzitní databázi a bude veřejně přístupná k nahlédnutí.

Jsem si vědom/a že, na moji bakalářskou práci se plně vztahuje zákon č. 121/2000 Sb., o právu autorském, o právech souvisejících s právem autorským a o změně některých zákonů, ve znění pozdějších předpisů, především ustanovení § 35 odst. 3 tohoto zákona, tj. o užití tohoto díla.

\qquad

\setlength{\dashlinedash}{1pt}
\setlength{\dashlinegap}{1pt}
\setlength{\arrayrulewidth}{1pt}

\noindent
\begin{tabularx}{\textwidth}{X r}
V Praze dne \today &
\begin{tabular}[b]{@{} p{6cm} @{}}
\\
\hdashline
\centering
podpis
\end{tabular}
\\
\end{tabularx}


\include{podekovani}

% === Nastavení obsahů pro dokumenty ===
% tabulka s obsahem
\tableofcontents

% seznam zkratek
% pouze u většího množství zkratek
%\include{seznamZkratek}

% seznam tabulek
% pouze pokud jsou použity více jak tři tabulky
%\newpage
%\listoftables
%\thispagestyle{empty}

% seznam obrázků
% pouze pokud jsou použity více jak tři obrázky
%\newpage
%\listoffigures
%\thispagestyle{empty}

\setcounter{page}{1} %cislo strany
\thispagestyle{empty}

\newpage

\pagestyle{fancy} %vlastni zahlavi/zapati
\renewcommand{\headrulewidth}{0.5pt} %bez linky v zahlavi
\renewcommand{\footrulewidth}{0pt} %linka v zapati
\lhead{\leftmark}       \chead{} \rhead{} %pole zahlavi (prazdna)
\lfoot{} \cfoot{} \rfoot{\thepage} %pole zapati


\section{Some section}
Text from input file. Here is some xyz in that file xyz xyz xyz xyz xyz.


%\value{graf}

% === Zdroje ===
\clearpage
\phantomsection % přidání odkazu do PDF záložek
\addcontentsline{toc}{section}{Seznam použitých zdrojů}
\renewcommand{\refname}{Seznam použitých zdrojů}

\printbibliography

\end{document}
